\documentclass[a4paper]{llncs}
\usepackage[plain]{algorithm}
\usepackage{listings}
%\usepackage{algorithmicx}
\usepackage{algpseudocode}
\usepackage{amsmath}
\usepackage{dcolumn}
%\lstset{language=C}
\usepackage{tikz}
\usetikzlibrary{positioning}
\usetikzlibrary{calc}
\usetikzlibrary{arrows,shapes,backgrounds}
\usetikzlibrary{shadows}

\lstdefinelanguage
  [GCC]{C++}
  []{C++}
{
  morekeywords={[2]__attribute__},%
  morekeywords={[3]vector_size,r,DeltaH}%
}
\lstset{
  language={[GCC]C++},%
  basicstyle=\small,%
  keywordstyle=\bfseries,%
  keywordstyle=[2]\bfseries,%
  keywordstyle=[3]\itshape,%
  %commentstyle=\color{gray},%
  stringstyle=\color{gray},%
  stepnumber=1,%
  numbersep=5pt,%
  tabsize=2,%
  captionpos=b,%
  breaklines=true,%
  breakatwhitespace=false,%
  showspaces=false,%
  showtabs=false,%
  columns=flexible%
}

%\usepackage{tikz}
\usetikzlibrary{positioning}
\usetikzlibrary{calc}
\usetikzlibrary{shadows}
\usetikzlibrary{fit}

\mode<presentation>
{
\usetheme{boxes}
\usecolortheme{crane}
%\usetheme{Warsaw}
\setbeamercovered{transparent}
}


\usepackage[polish]{babel}
\usepackage[T1]{fontenc}
\usepackage[utf8]{inputenc}
\usepackage{psfrag}
\usepackage{graphicx}
\usepackage{times}

\usepackage{listings}


\makeatletter
\newenvironment{minsizebox}[2]{%
    \pgfmathsetlength\@tempdima{#2}%
    \pgfmathsetlength\pgf@yc{\pgfkeysvalueof{/pgf/inner ysep}}%
    \advance\@tempdima by -2\pgf@yc
    \begin{lrbox}{\@tempboxa}%
        \begin{minipage}[t]{#1}%
           \vspace{0pt}%
}{%
        \end{minipage}%
    \end{lrbox}%
    \ifdim\@tempdima>\dp\@tempboxa
        \dp\@tempboxa=\@tempdima
    \fi
    \box\@tempboxa
}
\makeatother
\tikzset{onslide/.code args={<#1>#2}{%
  \only<#1>{\pgfkeysalso{#2}} % \pgfkeysalso doesn't change the path
}}

\newenvironment{pic}[1]{%
\begin{center}\begin{tikzpicture}#1}{%
\end{tikzpicture}
\end{center}
\end{lstlisting}}


\newcommand{\listing}{\node[listing]{\begin{lstlisting}
\end{lstlisting}
}
}


\tikzset{common/.style={fill=white,draw=black,rounded corners=1mm}}
\tikzset{listing/.style={fill=white,draw=black,font=\scriptsize,
rounded corners=0.5mm,drop shadow}}
\tikzset{title/.style={fill=white,font=\Large,rectangle, draw=black,
drop shadow, minimum width=\textheight, minimum height=15mm,
 rounded corners =3mm}}

\tikzset{display/.style={fill=white,rectangle, draw=black,
drop shadow, minimum width=8cm, minimum height=15mm,
 rounded corners =3mm}}

\renewcommand{\v}[1]{\vec{#1}}
\newcommand{\av}[1]{\langle#1\rangle}
\renewcommand{\a}[1]{\v{a}_{#1}}
\renewcommand{\L}{\mathcal{L}}
\renewcommand{\v}[1]{\vec{#1}}
\newcommand{\av}[1]{\langle#1\rangle}
\newcommand{\vphi}{\varphi}
\newcommand{\vi}{{\vec{i}}}
\newcommand{\vj}{{\vec{j}}}
\newcommand{\vmu}{\vec{\mu}}

\newcommand{\fillblockgray}[2]{
\pgfmathtruncatemacro\llx{\bksize*(#1)-2}
\pgfmathtruncatemacro\lly{\bksize*(#2)-2}
\pgfmathtruncatemacro\urx{\bksize*(#1+1)+1}
\pgfmathtruncatemacro\ury{\bksize*(#2+1)+1}
\foreach \x in {\llx, ..., \urx}
  \foreach \y in {\lly,...,\ury} {
    \fill[gray] (\x, \y) circle(.25);
  }
}

\newcommand{\npartition}[3]{
\pgfmathtruncatemacro\row{Mod(#2,4)}
\ifnum \row = 0
   \pgfmathtruncatemacro\num{Mod(#1,4)}
\else \ifnum \row  = 1
   \pgfmathtruncatemacro\num{Mod(#1,4)+4}
\else \ifnum \row  = 2
   \pgfmathtruncatemacro\num{Mod(Mod(#1,4)+2,4)}
\else \ifnum \row  = 3
   \pgfmathtruncatemacro\num{Mod(Mod(#1,4)+2,4)+4}
\fi
\fi
\fi
\fi
\ifnum \num = #3
\draw[draw=black,fill=red] (#1, #2) circle(0.4);
\fi

}

\newcommand{\markpartition}[3]{
\pgfmathtruncatemacro\llx{\bksize*(#1)}
\pgfmathtruncatemacro\lly{\bksize*(#2)}
\pgfmathtruncatemacro\urx{\bksize*(#1+1)-1}
\pgfmathtruncatemacro\ury{\bksize*(#2+1)-1}
\foreach \x in {\llx, ..., \urx}
  \foreach \y in {\lly,...,\ury} {
    \npartition{\x}{\y}{#3}
  }
}


\definecolor{colorone}{HTML}{777777} % default 116699
\definecolor{colortwo}{HTML}{DDDDDD} % default CCCCCC
\def\bksize{8}
\def\bkcount{4}
\def\lcsize{5}

\title{ GPU-accelerated and CPU SIMD Optimized \\
        Monte Carlo Simulation of $\phi^4$ Model }
\author{ Piotr Bialas\inst{1}\inst{2} \and Jakub Kowal\inst{1}
         \and Adam Strzelecki\inst{1} }
\institute{ Faculty of Physics, Astronomy and Applied Computer Science \\
            Jagiellonian University ul. Reymonta 4, 30-059 Krakow, Poland 
            \and Mark Kac Complex Systems Research Centre \\
            Faculty of Physics, Astronomy and Applied Computer Science \\
            Jagiellonian University, Reymonta 4, 30-059 Krakow, Poland }

\begin{document}

\maketitle

\begin{abstract}
  In this contribution we describe an efficient GPU implementation of
  the Monte-Carlo simulation of the Ginzburg--Landau model. We compare
  the performance with a parallelized and vectorized CPU code and
  discuss the observed differences. 
\end{abstract}

\section{Introduction}

The Graphical Processing Units (GPU) emerge this days as the most
efficient device for numerical calculation. To ascertain this it is
sufficient to look at the list of 500 supercomputers\cite{top500} and count how
many of them has GPU accelerators and the trend is still
increasing. This was made possible not only by the accessibility of
the hardware but also by the realization from the side of the vendors
that those units can be used also for computations, and providing user
with suitable software interface to do this. Technologies like CUDA
and OpenCL make the GPU relatively easy accessible to C/C++
programmers. In the meantime also most of the numerical packages
vendors have added GPU support to their tools.

However like with most of the computer languages there is a gap
between using them and using them efficiently. The GPU are not only
{\em massively parallel} machines but essentially {\em vector
  machines}. This is a trend visible also in the CPU market: the SSE extensions
introduced four floats wide vector (SIMD) instructions, AVX eight
floats wide and Xeon $\phi$ 16 floats wide. 

This contribution is concerned with an efficient implementation of the
Monte-Carlo simulations of the $\varphi^4$ or Ginzburg--Landau model.

\section{Ginzburg--Landau model and Monte-Carlo simulations}

Ginzburg--Landau model in its discretized form is defined by the Hamiltonian\cite{parisi}: 
\begin{equation*}\label{eq:hamiltonian}\begin{split}
H[\varphi]&=\sum_{\vi}\Biggl(
\frac{1}{2}\sum_{\mu=1}^d(\vphi(x_\vi+\a{\mu})-\vphi(x_\vi))^2\\
&\phantom{=\sum_{\vi}\bigl(}+\frac{\mu^2}{2}|\vphi(x_{\vi})|^2+
\frac{g}{24}(|\vphi(x_{\vi})|^2)^2\\
&\phantom{=\sum_{\vi}\bigl(}+
\frac{1}{2\Lambda}\Bigl(\sum_{\mu=1}^d
(\vphi(x_\vi+\a{\mu})-2\vphi(x_{\vi})+\vphi(x_\vi-\a\mu))\Bigr)^2\Biggr).
\end{split}
\end{equation*}
The
problem is defined as follows: having a vector field $\vphi$ defined
on a regular rectangular two or three dimensional grid we want to
generate the field configurations with probability proportional to
$\exp(-H(\vphi))$.  

The actual generation is done by the mean of the Metropolis
algorithm. This amounts to sequentially updating all the points of the
lattice. During each update we change the value of the field $\vphi$
in on a single lattice point ($\vphi_i$ is a shorthand for $\vphi(x_\vi)$).
\begin{equation}
\vphi_i\longrightarrow\widetilde{\vphi_i}=\vphi_i+\eta.
\end{equation}
The $\eta$ is a pseudo--random number drawn from a distribution such
that $P(\eta) = P(-\eta)$. In practice we draw the $\eta$ from the
uniform distribution on interval $[-\varepsilon,\varepsilon)$.  The we
calculate the difference of the Hamiltonian's
\begin{equation}
\Delta H=H[\widetilde{\vphi_i}]-H[\vphi_i]
\end{equation}
If $\Delta H < 0$ we accept the change and substitute
$\widetilde{\vphi_i}$ for $\vphi_i$, otherwise we accept the change
with the probability $\exp(-\Delta H)$. In other words we accept the
change with the probability
\begin{equation}
P(\vphi_i\rightarrow\vphi_i+\varepsilon)=\min\left(1,\exp(-\Delta H)\right)
\end{equation}
The crucial feature of this algorithm is that the update is local {\em
  i.e.} $\Delta H$ depends only on the values of fields in the
immediate neighborhood of the updated point 
\begin{equation}
\begin{split}
\Delta H & = -(\widetilde{\vphi_i}-\vphi_i)
\underbrace{\left(c^{01}_i-\frac{1}{\Lambda}(c^{02}_i-4 d c^{01}_i+2 c^{11}_i )\right)}_{c_i}\\ 
&\phantom{= -}
+(\widetilde{\vphi}^2_i-\vphi_i^2)\left(d+\frac{\mu^2}{2}+d (1+2d)\frac{1}{\Lambda}\right)\\
&\phantom{= -}+\frac{g}{4!}\left(\widetilde{\vphi}^4_i-\vphi^4_i\right)
\end{split}
\end{equation}
where
\begin{equation}\begin{split} 
c_i^{01}&=\sum_{\mu=1}^d\left(\vphi(x_\vi+\a\mu)+\vphi(x_\vi-\a\mu)\right)\\
c_i^{02}&=\sum_{\mu=1}^d\left(\vphi(x_\vi+2\a\mu)+\vphi(x_\vi-2\a\mu)\right)\\
c_i^{11}&=\sum_{\mu=}^d\sum_{\nu=1}^{\mu-1}
\bigl(\vphi(x_\vi+\a\mu+\a\nu)+\vphi(x_\vi-\a\mu-\a\nu)\\
&\phantom{+=\sum_{\mu=}^d\sum_{\nu=1}^{\mu-1}
}
+ \vphi(x_\vi+\a\mu-\a\nu)+\vphi(x_\vi-\a\mu+\a\nu)
\bigr)
\end{split}
\end{equation}
The term $c_i$ is referred to as {\em corona}. 
In our case, because of
the later term in \eqref{eq:hamiltonian}, this neighborhood is
extended compared to usual nearest neighbors (see figure~\ref{fig:nn}). 


\section{GPU implementation}

While model is inherently parallelizable, grid points that lie in the
same neighborhood cannot be updated together. Taking into account a
larger neighborhood means that a simple checkerboard decomposition
pattern cannot be used and we have devised a new grid decomposition
scheme. We choose the decomposition depicted in the
figure~\ref{fig:part}. In 2D we need eight sublattices and 16
sublattices in 3D.
\begin{figure}
\begin{center}
\begin{tikzpicture} 
%\pgfmathtruncatemacro\lcsize{\bksize * \bkcount}
\def\lcsize{8}
\pgfmathtruncatemacro\lcsizem{\lcsize - 1}
\pgfmathtruncatemacro\lcsizeb{\lcsize - \bksize}
\pgfmathtruncatemacro\bksizea{\bksize - 2}
\pgfmathtruncatemacro\bksizeb{\bksize - 1}
\pgfmathtruncatemacro\bksizec{\bksize * 2}
\pgfmathtruncatemacro\bksizem{\bksizec - 1}
\pgfmathtruncatemacro\bksized{\bksizec + 1}
\draw[very thin, gray] (-.5, -.5) grid (\bksize - .5, \bksize - .5);
\foreach \x in {0,...,\lcsizem} 
\foreach \y in {0, ...,\lcsizem}
\draw[fill=white,draw=black]  (\x, \y) circle[radius=0.1];

\coordinate (c) at (4,4);

\node at (c)  [circle,fill=red,draw=black] {};

\def\coronaa{(0,1), (0,-1), (1,0), (-1,0)}
\def\coronab{(0,2), (0,-2), (2,0), (-2,0)}
\def\coronac{(1,1), (1,-1), (-1,-1), (-1,1)}

\foreach \p in \coronaa {
\path (c) +\p node  [rectangle,fill=gray, draw=black]    {};
}

\foreach \p in \coronab {
\path (c) +\p node  [diamond,fill=gray,draw=black]    {};
}

\foreach \p in \coronac {
\path (c) +\p node  [circle,fill=gray, draw= black]    {};
}

\end{tikzpicture}
\end{center}
\caption{\label{fig:nn}The neighborhood used to calculate the $\Delta H$:
$c^{01}$ -- squares, $c^{11}$ -- circles and $c^{02}$ -- diamonds.}
\end{figure}

\begin{figure}
\begin{center} 
\begin{tikzpicture}[scale=0.95]

% \clip[rounded corners=0cm] (-1, -1) rectangle (\bksize + 1, \bksize + 1);

% \draw[help lines] (-1, -1) grid (\bksize + 1, \bksize + 1);

\pgfmathtruncatemacro\bksizem{\bksize - 1}

% % lattice points
 \foreach \x in {0, ..., \bksizem}
   \foreach \y in {0, ..., \bksizem} {
     \pgfmathtruncatemacro\four{Mod(\y,4)}
     \pgfmathtruncatemacro\eight{Mod(\x,4)+4*Mod(\y,2)}
     \ifnum \four<2
     \pgfmathtruncatemacro\eight{Mod(\x,4)+4*Mod(\y,2)}
     \node[draw,rectangle,minimum size=0.6cm] at (\x,\y)  {\eight};
     \else
     \pgfmathtruncatemacro\eight{Mod(\x+2,4)+4*Mod(\y,2)}
     \node[draw,rectangle,minimum size=0.6cm] at (\x,\y)  {\eight};
     \fi
}
% neighborhood
\def\neighborhood{--
  ++( 0,  -.5)-- ++( 1,  0)-- ++( 0, -.5 )-- ++(0, -.5)-- ++( 1, 0  )--
  ++( 0, -1.0)-- ++( 1,  0)-- ++( 0,  1.0)-- ++(1, 0  )-- ++( 0, 1.0)--
  ++( 1,  0  )-- ++( 0,  1)-- ++(-1,  0  )-- ++(0, 1  )-- ++(-1, 0  )--
  ++( 0,  1  )-- ++(-1,  0)-- ++( 0, -1  )--
  ++(-1,  0  )-- ++( 0, -1)-- ++(-1,  0  )-- ++(0, -.6)}

\pgfmathtruncatemacro\lchalf{\bksize / 2}


% main center neighborhood
\path[draw=red,  rounded corners=.1cm, shorten >=2pt, line width=.75mm]
(\lchalf - 2.5, \lchalf) \neighborhood;
\end{tikzpicture}
\end{center}
\caption{\label{fig:part}Partition of the lattice into independent
  sublattices. Squares with same number can be updated in parallel.}
\end{figure}



The GPU is no exception as to what concerns the speed of the memory
access.  While very fast, the throughput from global memory is still
almost two orders of magnitude slower then needed to sustain the
maximal flops output.  The Nvidia cards contain so called { Scalar
  Multiprocessors} (SM). On Fermi architecture each SM is equipped
with up 48KB shared memory and 32K of 32bit registers\cite{Fermi}
which act as user managed cache. While starting from the compute
capability 2.0 Nvidia cards are also equipped with L1 and L2 caches,
the use of automatic cache is, as we will show later, no advantageous
in terms of performance.

To efficiently use the shared memory  we adopt the hierarchical scheme from
ref.~\cite{weigel} suitably modified to account for bigger
neighborhood.  We first divide the whole lattice in blocks of
$B_x\times B_y$ points (see figure~\ref{fig:blocks}). 
\begin{figure} 
\begin{center}

\begin{tikzpicture}[scale=0.35]

\pgfmathtruncatemacro\lcsize{\bksize * \bkcount}
\pgfmathtruncatemacro\lcsizem{\lcsize - 1}
\pgfmathtruncatemacro\lcsizeb{\lcsize - \bksize}


\draw[very thin, gray] (-.5, -.5) grid (\lcsize - .5, \lcsize - .5);

\foreach \x in {0, \bksize, ..., \lcsizeb}
  \foreach \y in {0, \bksize, ..., \lcsizeb} {
    \pgfmathtruncatemacro\n{Mod(\x/\bksize, 2)+2 * Mod(\y/\bksize, 2)+1}
    \draw[colorone, xshift=\x cm, yshift=\y cm, rounded corners=2pt]
      (-.4, -.4) rectangle
      node[red, fill opacity=0, anchor=center, font=\Huge\bfseries\sffamily]
      {\n} (\bksize - .6, \bksize - .6);
  }

\foreach \x in {0, ..., \lcsizem}
  \foreach \y in {0, ..., \lcsizem}{
 
    \path[draw=colorone, fill=white] (\x, \y) circle(.25);
}


\fillblockgray{0}{0}
\markpartition{0}{0}{0}
\fillblockgray{0}{2}
\markpartition{0}{2}{0}
\fillblockgray{2}{0}
\markpartition{2}{0}{0}
\fillblockgray{2}{2}
\markpartition{2}{2}{0}
\end{tikzpicture}
\end{center}
\caption{\label{fig:blocks}Partition of the lattice into
  blocks.  The sites 
marked in gray are loaded into shared memory.}
\end{figure}
To process each point inside a block we also need the neighboring
points.  We divide the grid of blocks into subgrids in such a way,
that blocks together with the neighborhood in each subgrid do not
overlap (see gray points in figure~\ref{fig:blocks}). In 2D we need
four of such subgrids and eight in 3D case.

We then start a kernel that process all blocks in one subgrid.  Each
block in this subgrid is assigned to a block of $N_{th}$ threads (CTA
-- Cooperative Threads Array). First we fetch the values of the fields
from global to shared memory (including border points see
figure~\ref{fig:blocks}).  After that each thread updates one site
from the first sublattice. This is done $N_{hit}$ times resulting in
what is know as {\em multi-hit} metropolis algorithm. The advantage is
that the corona $c_i$ does not need to be recalculated agin and most
of the needed variables resides already in registers.  Then after
synchronization, next sublattice is updated and so on. This whole
procedure is repeated $N_{loc}$ times, again taking advantage of
the fact that the field values are already in the shared memory. After
that the kernel writes the shared memory back into global and new
kernel is started processing next subgrid of blocks (see
algorithm~\ref{alg:gpu}).


From this description Each threads in CTA is updating 8
lattice sites in 2D, so one CTA is updating $8 N_{th}/(B_x B_y)$ grid
blocks ($16 N_{th}/(B_x B_y B_z)$ in 3D). Taking into account the number of
subgrids one thread is updating $8*4=32$ lattice sites in 2D and
$16\times 8=128$ in 3D.

Monte-Carlo calculations require a good source of random numbers. We
use the Tausworthe pseudo--random number
generator\cite{howes_thomas07}. The performance could be increased by
the use of simpler generator, but we have not attempted this in this
work.

\begin{algorithm}
\begin{algorithmic}[1]
\For{every global sweep}
\For{every block subgrid}
\For{every block in subgrid}
\State load block  into shared memory
\For{every local sublattice}
\For{every site $i$ in local sublattice}
\State calculate corona $c_i$ 
\For{ k = 1 ..  $N_{hit}$ }
\State $\widetilde{\vphi}_i \gets \vphi_i+\eta$
\State $\Delta H \gets H[\widetilde{\vphi}_i]-H[\vphi_i]$
\If{$\exp(-\Delta H) > r$}
\State $\vphi_i \gets \widetilde{\vphi}_i$
\EndIf
\EndFor
\EndFor 
\State {\tt \_\_syncthreads();}
\EndFor
\EndFor
\EndFor
\EndFor
\end{algorithmic}
\caption{\label{alg:gpu}}
\end{algorithm}


One of the problems with lattice simulations is the treatment of the
boundary conditions. The algorithm requires access to the site
neighbors during each update. The index of neighbors can be calculated
using simple arithmetics, except at the boundaries. The indexing of
neighbors requires then checking for the boundary at each update or
expensive modulo operations. Another technique is to use the
precalculated and stored indices: for every site we store the index of
four or six nearest neighbors. This is also very flexible but
requires a large amount of memory. This can have a negative impact on
the performance and is not well suitable for the GPU which usually
have less RAM then CPU's. In our algorithm the checking for boundary
is needed only when loading to shared memory. Then we can use simple
index arithmetics.




\section{CPU implementation}
% CPU implementation
To make a fair comparison with the CPU we have to use both multicore
as vector (SSE/AVX) programming.  The parallelisation on many cores
relatively easy using the OpenMP. The same restriction apply as on
GPU, so we partition the the lattice in the same way, but we use only
one level {\em i.e.} we do not partition the lattice into blocks.  The
SIMD instructions are used to process four (SSE) or eight (AVX)
updates in parallel.  At present only portable way to use vector
instructions is to use compiler intrinsics\cite{intr}. We have defined
our own vector type that uses combination of the 
 \lstinline!__attribute__((vector_size(N)))!  directive and intrinsics
to provide a drop--in replacement for scalar types.

Unfortunately not all scalar \emph{x86} instructions have vector counterparts,
which is crucial requirement for vectoring scalar code. In particular
\emph{x86} misses direct vector \emph{XMM} gather and scatter instructions
(load/save vector register data from/to memory indexed by other vector
register), unlike GPU where gather and scatter is key feature. Another drawback
in lack of AVX 256-bit integer operations, which makes random number generator
utilize 128-bit SSE only, even on 256-bit AVX capable CPUs.

In our CPU implementation we use a small ``trick''. Instead of using the
\begin{lstlisting}
        if ( exp(-DeltaH) > r )
\end{lstlisting}
condition for the Metropolis algorithm ($r$ is a pseudo--random number) we use
an equivalent condition
\begin{lstlisting}
        if ( -DeltaH >  log(r) )
\end{lstlisting}

The advantage is now that the argument of $\log$ is independent of
$\Delta H$ and can be calculated beforehand allowing to hide some of
the latency. This implementation is safe even in case when $r$ is
equal zero as $\log$ then returns {\tt -Inf}.  The calculations of the
four or eight $\log$'s in parallel is done using the Intel's svml
library.

As already mentioned in the previous section indexing the neighbors
can be quite costly operation requiring expensive modulo
operations or large amount of memory. However if
we restrict ourself to the lattice dimensions the are powers of two
the nearest neighbor calculations can be made using addition and {\tt
  and} operations. This is quite a restrictive approach, not suitable
for real applications but we have decided to use it for comparison
with the GPU version. This is in some respect the ``upper bound'' of
the CPU performance.


\section{Performance}
\label{sec:performance}


The overall performance of the algorithm expressed in Gflops or
updates per sec. will depend strongly on the values of the $N_{loc.}$
and $N_{hit}$. The optimal values of those parameters will depends on
the physics of the model in particular on the values of parameters
$\mu^2$, $g$ and $\Lambda$. To describe the performance in a general
way we have decided to fit the execution time $T$ to the following
model:
\begin{equation}\label{eq:model}
T  = a+ V \cdot b +
N_{glob.} V \left(c + N_{loc.}\left(d + N_{hit}\cdot e\right)\right) . 
\end{equation}
where $V$ is the volume of the lattice.  The interpretation of the
parameters of the model is given in table~\ref{tab:pars-int}.

\begin{table}
\begin{center} \begin{tabular}{lcp{8cm}p{2cm}}
$c$ &--& time per spin needed to load it to shared memory and back.&\\
$d$ &--& time per spin needed to calculate the corona.& 26 nops\\
$e$ &--& time per spin needed to for update including calculation of two
random numbers and exponential. & 64 nops + exp
\end{tabular}
\end{center}
\caption{\label{tab:pars-int}Interpretation of the parameters in the
model~\ref{eq:model}. The last column gives the number of numerical operations
performed by the corresponding step of the algorithm.}
\end{table}

\begin{table}
\begin{center} 
\begin{tabular}{|p{4cm}|l|l|l|}\hline\hline
 & float & int  & bit \\\hline
coronas & 26 & & \\\hline
$\Delta H  $ & 13  & & \\\hline
PRNG & 1 & 2  & 21  \\\hline
$\widetilde{\vphi}_i \gets \vphi_i+\eta$& 3 & &\\\hline\hline
\end{tabular}
\end{center}
\caption{\label{tab:instr-count}}Number of instruction in different steps of the
algorithm. float -- floating point, int -- integer arithmetic, bit -- shifts and
bitwise logical. In the last row we do not include the call to PRNG. 
\end{table}

The number of updates is given by
\begin{equation}
N_{up}=V\cdot N_{glob.}\cdot N_{loc.}\cdot N_{hit}
\end{equation}
so the time per update is given by
\begin{equation}\begin{split}
\frac{T}{N_{up}}=&\frac{a+ V \cdot b + N_{glob.} V \left(c + N_{loc.}\left(d + N_{hit}\cdot e\right)\right)}{V\cdot N_{glob.}\cdot N_{loc.}\cdot N_{hit}}\\
\approx&
\frac{ \left(c + N_{loc.}\left(d + N_{hit}\cdot e\right)\right)}{N_{loc.}\cdot N_{hit}}
\end{split}
\end{equation}
Using the data from the table~\ref{tab:pars-int} we obtain that the number of the operations performed is
\begin{equation}
N_{up}=V\cdot N_{glob.}\cdot N_{loc.}(26+\cdot N_{hit}(64+exp))
\end{equation}
\begin{equation}
nops\approx \frac{\cdot N_{loc.}(26+\cdot N_{hit}(64+exp))}{ \left(c + N_{loc.}\left(d + N_{hit}\cdot e\right)\right)}
\end{equation}
We have gather the data on the execution times of the 3D program for
three different lattice sizes. First thing we noticed is that it is
not possible to fit all the data simultaneously to the model
\eqref{eq:model}. However when fitting data for each lattice size
separately we obtain a very good fit accuracy better then $1\%$ on the
average. The results of the fit are included on the table~\ref{tab:fit}.

\begin{table}
\begin{center}
\begin{tabular}{||r|D{.}{.}{8}|D{.}{.}{8}|D{.}{.}{8}|D{.}{.}{8}||}
\hline\hline
\multicolumn{1}{|c}{$L^3$} 
&\multicolumn{1}{|c}{$a+ L^3b$} 
&\multicolumn{1}{|c}{$c$} 
&\multicolumn{1}{|c}{$d$} 
&\multicolumn{1}{|c|}{$e$}\\\hline\hline
GPU &\multicolumn{4}{c||}{} \\\hline
$64^3$  & 0.56  & 1.64e-9       & 	2.76e-10       & 1.78e-10 \\
$128^3$ & 0.49  & 1.06e-9       &	2.06e-10       & 1.35e-10 \\
$256^3$ & 0.41  & 9.01e-10	&       1.85e-10       & 1.22e-10\\
$512^3$ & 0.86  & 6.61e-10	&       2.02e-10       & 1.10e-10\\\hline\hline
CPU &\multicolumn{4}{c||}{} \\\hline
$64^3$  &  0.12 &       & 	1.27e-8 &	1.51e-9\\
$128^3$ & 1.17  &       &	1.34e-8 & 	1.45e-9\\
$256^3$ & 0.93  &       &	1.26e-8 &       1.43e-9\\
$512^3$ & 3.72  &       &       1.35e-8 &       1.44e-9\\\hline\hline
\end{tabular}
\end{center}
\caption{\label{tab:fit}Results of the fit of the formula \eqref{eq:model} GPU
and \eqref{eq:model-cpu} CPU.}
\end{table}
In case of CPU as we do not have the shared memory the formula for the fit is
\begin{equation}\label{eq:model-cpu}
T  = a+ V \cdot b +
N_{glob.} V \left(d + N_{hit}\cdot e\right) . 
\end{equation}
The coefficient $d$ is the time per spin needed to load it from and to memory
and the calculation of the corona. the time per update in this case is given by:
\begin{equation}\begin{split}
\frac{T}{N_{up}}=&\frac{a+ V \cdot b + N_{glob.} V \left(d + N_{hit}\cdot e\right)}{V\cdot N_{glob.}\cdot N_{hit}}\\
\approx&
\frac{d + N_{hit}\cdot e}{\cdot N_{hit}}
\end{split}
\end{equation}
and
\begin{equation}
nops\approx \frac{ 26+N_{hit}(64+exp)}{ \left(d + N_{hit}\cdot e\right)}
\end{equation}
\begin{table}
\begin{center}
\begin{tabular}{|r|rrr|rrr|c|}
\hline\hline
\multicolumn{1}{|}{} & \multicolumn{3}{c|}{GPU}
                     & \multicolumn{3}{c|}{CPU}& GPU/CPU\\\hline
\multicolumn{1}{|c}{size} & 
\multicolumn{1}{|c|}{ns} &
 \multicolumn{1}{c|}{Gops} &
 \multicolumn{1}{c|}{Gexp/s} & 
 \multicolumn{1}{c|}{ns} &
\multicolumn{1}{|c}{Gops} & \multicolumn{1}{|c|}{Glog/s} &\\\hline\hline
$128^3$ & 0.19 & 368 & 5.2   &    4.8 & 14.6 & 0.21 & 25.1\\\hline
$256^3$ & 0.17 & 408 & 5.8   &    4.6 & 15.4 & 0.22 & 26.5\\\hline
$512^3$ & 0.16 & 430 & 6.1   &    4.8 & 14.6 & 0.21 & 29.5\\\hline\hline
\end{tabular}
\end{center}
\caption{\label{tab:comp} Performance of the algorithm.}
\end{table}

\subsection{GPU cache}

The good performance of the presented GPU algorithm comes from the effective
use of the shared memory. We have also checked to which extend this could be
achieved by just using the automatic cache capabilities of modern GPU's. To
this end we have implemented a version of the algorithm which did not use the
shared memory, however the access pattern was the same. The idea was that the
first iteration of the local sweep loop would fill the L1 cache and consecutive
iterations would not need to access the global memory and the performance
should be comparable to our original implementation.

This turned out not to be the case. The algorithm using the L1 cache
was approximately six times slower then the one using shared
memory. This deteriorated performance could be traced back to large
number of cache misses. Nominally the $20^3$ block should fit in L1
cache as it did fit into the shared memory. However the cache line is
32 words long and the blocks are not aligned on the cache lines
boundaries (see gray points on the figure~\ref{fig:blocks}). On top of
that the cache is probably not fully associative. All that together
implies that the cache size should be up to four or five times bigger
to accommodate the block. 


\section{Discussion}

Results prrsented in table~\ref{tab:comp} show clear advantage of the
GPU over CPU. This is hardly suprising, but the $26\times$ difference
is much higher that comes from comparison of theoretical performace of
tested i5 CPU to \emph{GTX 470}. The GTX is thus more efficient in
that sense that is capable of achieving performance much closer to the
peak. This can be attributed two two main reasons:
\begin{itemize}
\item The shared memory that acts as an managed cache allows to use
  the small cache size much more effectively. In particular it is
  possible on the GPU to fit the whole $16^3$ grid block into the
  shared memory. This is not possible using the conventional cache
  both on GPU and CPU leading to large number of cache misses and
  decrease of performance.
\item Modern CPU are not full vector processors. Not every scalar
  instruction has a vector counterpart. In particular the eight words
  wide integer instructions are missing from the AVX instruction set.
  This makes impossible to make a 256-bit pseudo-random number
  generator.
\item Efficient load and store vector (gather/scatter) instruction are
  also missing from CPU.  When not loading/storing form the
  consecutive memory location they are efectively serialised into
  scalar loads/stores.
\end{itemize}

\begin{thebibliography}{9}
\bibitem{top500}{\tt http://www.top500.org/}
\bibitem{parisi} G.~Parisi ``Statistical Field Theory'' Chapter 5, Perseus Books Publishing (1998).
\bibitem{Fermi} 
``NVIDIA's Next Generation CUDA ComputeArchitecture: Fermi'' \\
\verb!http://www.nvidia.com/content/PDF/fermi_white_papers/!\\
\verb!NVIDIA_Fermi_Compute_Architecture_Whitepaper.pdf!
\bibitem{weigel} M.~Weigel, J. Comput. Phys. \textbf{231}, 3064 (2012).
\bibitem{howes_thomas07}
Lee Howes and David Thomas.
``{\em Efficient random number generation and application using {CUDA}.}''
In Hubert Nguyen, editor, {\em GPU Gems 3}, chapter~37. Addison
  Wesley, August 2007.
\bibitem{intr} ``Intel C++ Intrinsic Reference.'' \verb!http://software.intel.com/sites/default!\\\verb!/files/m/9/4/c/8/e/18072-347603.pdf! \\
Intel Intrinsics Guide\\ \verb!http://software.intel.com/en-us/articles/intel-intrinsics-guide!

\end{thebibliography}

\end{document}
