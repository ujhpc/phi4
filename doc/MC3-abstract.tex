\documentclass[11pt,a4paper]{article}
 
\usepackage[utf8]{inputenc}

% Page Format
\topmargin      = 0pt
\oddsidemargin  = 0pt
\evensidemargin = 0pt
\parindent      = 0pt
\parskip        = 5pt
\textheight     = 220truemm
\textwidth      = 170truemm

\begin{document}

\begin{samepage}
\begin{center}
{\Large{\bf GPU-accelerated and CPU SIMD Optimized \\ Monte Carlo Simulations of $\phi^4$ Model }}

\vspace {5mm}
{\bf P. Białas, J. Kowal, A. Strzelecki}\\
{\em Faculty of Physics, Astronomy and Applied Computer Science, \\ Jagiellonian University in Krakow }\\

\end{center}
\end{samepage}

{\bf Abstract.}  We present a high-performance implementation of
Monte-Carlo simulations of 2 and 3 dimensional $\varphi^4$ model
both on GPU and on CPU using \emph{OpenMP} and \emph{SSE/AVX} SIMD
vectorization for \emph{Intel Core i7}. This simulation is a part of
on ongoing educational project but we took this opportunity to compare
the two programming models and check the claims of over $100\times$
advantage of GPU over generic CPU made recently in the literature.

The implemented model differs from the ones presented in the
literature ({\em e.g.} \cite{Preis20094468}) by the extended
neighborhood that has to be taken into account when updating a single
lattice field. While model is inherently parallelizable, grid points
that lie in the same neighborhood cannot be updated together. That
proved to be a problem especially on the GPU where much more threads
access the grid in parallel.  Taking into account a larger
neighborhood means that a simple checkerboard decomposition pattern
cannot be used and we have devised a new two-level grid decomposition
scheme.  Greatest challenge for GPU implementation was to provide
efficient memory synchronization satisfying those constraint.  Shared
memory was used extensively to minimize update latencies between
single threads.  The CPU implementation have run into similar problems
although less severe because of smaller number of parallel
threads. We have however opted to use the same partitioning scheme as
on the GPU.

To ensure valid simulation results we had to provide efficient random
number generator with good statistical quality and long period. We
have chosen \emph{Tausworthe} random generator\cite{LEcuyer96} as the
best one providing uniform distribution for the simulation. Our GPU
\emph{CUDA} implementation is based on \emph{GPU Gems
  3}\cite{howes_thomas07}, while our own CPU implementation uses
\emph{SSE2} SIMD integer extensions.

Altogether we managed to achieve $0.13$ nanoseconds for single lattice
field update on \emph{NVIDIA GTX 470}, reaching around $430$ Gflops of
$1088$ Gflops peak performance of this device. 


While SSE gives theoretical $4\times$ benefit for single precision
floating point operations, AVX providing $8\times$, not all scalar
\emph{x86} instructions, have vector counterparts.  In particular
direct \emph{XMM} registers gather and scatter and vectorized integer
operation for full length AVX 256-bit registers are missing, which makes
impossible to port random number generator from 128-bit SSE to 256-bit
AVX. Initially planned for AVX standard, these were postponed to AVX2
planned for 2013. As soon AVX2 capable CPU devices appear on the market
we plan to revise our evaluation.


Our CPU \emph{OpenMP} and \emph{SSE/AVX} implementation compiled via
GCC 4.4 or higher and running on \emph{Intel Core i5 2.5 Ghz} quad
core CPU presented $15\times$ performance boost comparing to single
threaded scalar code and $3.76$ nanoseconds for single lattice field
update.  Which gives the $15$ Gflops. There is no significant increase
in performance while switching from SSE to AVX instructions.

\newpage

This gives around $28\times$ advantage to GPU, which is noticeably
less than promised by many publications, however much higher that
comes from comparison of $160$ Gflops peak performance of tested i5
CPU to \emph{GTX 470}\cite{Lee:2010:DGV:1816038.1816021}. This can be
traced back to 128-bit only \emph{Tausworthe} random number generator
implementation and inefficient store and load operations
(gather/scatter) which can be partially mitigated by the reorganizing
the way in which the grid is stored in the memory. This is the subject
of an ongoing work.

\bibliography{MC3-abstract}{}
\bibliographystyle{plain}

\end{document}
